%%------------------------------------------------------
%  
%  Implementation include for dissertation
%
%------------------------------------------------------


\subsection{Technical Challenges}

\subsubsection{ Sourcing Scientific Papers}
For Partridge to be a success, it is important to include a large variety of
different scientific papers in its corpus. Therefore, several sources of
scientific papers were explored and considered.

The most convenient and accessible source is the ART corpus that SAPIENTA was
trained with\cite{citeulike:11077287}. This corpus is already stored using the
CoreSC schema and has been pre-annotated. This means that Partridge would not
need to do any conversion or pre-processing on the papers. These scientific
papers are all chemistry papers. Therefore, more papers from other scientific
domains are needed to get a wider variety of topics and make Partridge useful
to as wide an audience as possible. 

The `mega-journals' PLOSOne \url{http://www.plosone.org/} and arXiv
\url{http://www.arxiv.org/} were suggested as sources for more open access
articles that could be added to Partridge. Both of these sites were found to
contain large volumes of open access papers. Most of the journals stored on the
arXiv site were in PDF format which, as discussed below, are difficult to
convert and annotate. However, PLOSOne use the SciXML markup language for
papers published through their journal, which made converting and annotating
them a lot simpler for Patridge. PLOSOne also publish all of their papers under
the Creative Commons Share-Alike-By-Attribution license\cite{ccbyattr}. This
meant that as long as the author information was left intact, all of the papers
could be used for data mining purposes.

\subsubsection{Acquiring Scientific Papers}
Once PLOSOne was identified as a suitable source of papers, a method for
acquiring large batches of papers to be processed was formulated.

\subsubsection{PDF Conversion}
Most scientific papers available on the internet are formatted as PDF
documents. However, Partridge uses and stores documents that use the CoreSC
schema by Soldatova and Liakata\cite{liakata2008guidelines}. Therefore some
spike work was carried out to determine the feasibility of converting papers
published as PDF documents into XML documents. Townsendi \emph{et al.} (2009) liken converting
PDF to XML to ``converting hamburgers into cows," they go on to explain that
PDF documents do not contain any semantic data and documents lose much of their
explicit structure when they are formatted in this way \cite{Townsend2009}.
Therefore, to convert PDF documents into an NLP-friendly format, some
heuristics must be used to detect the document's structure\cite{pdfminer}.

This was the first big challenge in the project. A prototype script was written
using a Python PDF extraction library called PDFMiner
(\url{http://www.unixuser.org/~euske/python/pdfminer/index.html}).  This
toolkit already contains some heuristics about how to extract text from PDF
documents, grouping together characters that appear very close to each other,
and separating paragraphs and headings when a larger area of whitespace is
detected\cite{pdfminer}. Despite these rules, the library still produced some
extraneous whitespace and newline characters as part of the output. A
subroutine to trim whitespace and newlines was added to the script to resolve
this problem. 

The next stage was to split the text into sentences in preparation for
processing with SAPIENTA. With the assistance of the NLTK library, a sentence
splitting subroutine was implemented. This used a machine learning algorithm
that had been trained to recognise sentence boundaries to split the text. Each
sentence was then added to a CoreSC compatible XML document for processing by
SAPIENTA.

Initially, the PDF conversion subroutine had a very high error rate due to the
variation in the formatting of scientific papers. It was suggested that PDFX
(\url{http://pdfx.cs.man.ac.uk/}), a free service hosted by the University of
Manchester could be used instead of PDFMiner for the initial PDF data
extraction. The main advantage of PDFX over the PDFMiner library is that it is a
trained machine learning system that has been trained using a large full-text
selection of scientific articles; PDFMiner uses more general heuristics
designed to process a large selection of different types of PDF document.

PDFX also provides output that already has some metadata, such as title,
author, and abstract, associated with it. PDFMiner did not provide any
metadata, and it was necessary for the script to guess which passage of text
was the abstract after the initial text extraction stage.

With the new PDF extraction method in place, the script ran without the need
to modify either of the whitespace sanitiser or sentence splitter routines. The
process was much more successful and able to produce SAPIENTA-compatible
documents from most of the PDF input files that were provided.

\subsubsection{Pre-processing with SAPIENTA}

With a successful PDF conversion script, the next step was to try and run
SAPIENTA over the converted papers and annotate them, ready for inclusion in
the Partridge corpus.

By default, SAPIENTA is packaged as a web-based tool, written in Java, that
can be downloaded (from \url{http://www.sapientaproject.com/software}) and used
to annotate one paper at a time. Dr Liakata was able to provide information on
two alternative ways of using the system. One method was to submit a remote
procedure call (RPC) to a server running SAPIENTA with a batch of papers and
retrieve the output. The other method was to use an alternative version of the
code that runs locally in a Python environment and could be modified to process
papers as a batch.

A script was written to send un-annotated XML documents to the remote SAPIENTA
server and retrieve a list of annotations. This worked well until the server
stopped replying to requests. This meant that no further conversions could be
carried out until the server was repaired and raised concerns about how the
remote servers might cope with a large number of automated requests from a
full version of Partridge.

The Python version of SAPIENTA was then downloaded and a test executed.
Unfortunately there were several data files missing from the package that had
to be acquired from Dr Liakata. 

SAPIENTA for Python also relies upon a package called CRFSuite which implements
Conditional Random Fields, a method for segmenting and labelling sequence
data\cite{CRFsuite}. This library did not compile properly on the test
environment and its creator had to be contacted via a mailing list (See
Appendix \ref{sec:crfemail}. After a few days, the owner responded and the
library was compiled successfully.  

Once all the data files and libraries were successfully in place, the Python
version of SAPIENTA was used to process some of the papers converted from PDF.
This was a great success and the annotations provided by SAPIENTA seemed to be
accurate. 



