Chosing an effective development methodology is vital to the
success or failure of a software project. A number of existing development
methodologies have been developed and used very effectively over the last 50
years. However, custom methodologies are also commonly used for projects that
aren't suited to one of the commonly used approaches. Some common approaches to
software development and the formulation of a custom methodology for Partridge
are discussed below.

\section{Existing Methodologies} 

Under the traditional `Waterfall' Software Development model, Requirements
Gathering, Analysis, Program Design, Coding, Testing and Operations were all
defined as formal phases in the development cycle. There is little flexibility
other than moving back up the waterfall to rectify mistakes after
testing\cite{Royce:1987:MDL:41765.41801}. This model is very focused on
paperwork and bureaucracy, trying to maintain a paper trail and manage risk
through accountability \emph{(Ibed)}. This approach to software development is
very heavyweight and slow and often produced software that did not match the
users' needs as a result \cite{Boehm1988}.

As an alternative to the heavyweight Waterfall approach, Beck \emph{et al.}
came up with the principle of the Agile Manifesto, favouring a lightweight and
responsive development model over the heavyweight, slow waterfall
system\cite{beck2001agile}. Many of Beck's ideas focus around working in a team
of developers and prioriting communication between team members \emph{(Ibid.)}.
This is most prominent in the Extreme Programming (XP) method of software
development. Since developing Partridge was an individual project, XP was not
really applicable. However, some concepts like rapid prototyping/spike work and
iterative release cycles were used as part of the Partridge development
methodology.

\section{ Partridge's Development Methodology}

A custom methodology was used for keeping track of development within
Partridge.  All design and planning documentation have been written up and
placed on a wiki which is accessible and modifiable by the author and both
supervisors. This creates a paper trail for all tasks and also allows
collaboration between involved parties through the Internet.

Throughout the project, weekly meetings were held with both project
supervisors. The notes from the preceeding week were analysed and each task
discussed in depth. New tasks were then noted down along with any observations
that should be documented. These were then uploaded to the wiki the following
day or earlier. Each party present at the meeting adds their own observations
to the notes page. This page was then reviewed at the next meeting.

Partridge made use of an agile, `iterative' development process. A working
version of the system was released at the end of each month. Tasks to be
undertaken in each iteration were stored in GitHub's issue manager program.
Priority was given to those tasks that provided the most value for the least
time investment. Test users were also allowed to submit bug reports to the
tracker system, and these were given priority if they were serious.

Programming tasks were added to the GitHub project tracker facility in the
same place as the Wiki was hosted. The `milestones' feature, which allows tasks
to be split into groups that must be completed by an arbitrary date, was used
to plan iterations and could be used to provide precise summaries of how the
project was progressing. In January, the project ran behind by a few weeks.
The time tracker explained that the project was precisely 17 days behind and
highlighted the set of tasks that needed to be completed to bring the project
back up to speed.

\subsection{ Initial Plan and Project Timeline }

During the early stages of the Partridge project, an initial system design was
drawn up and a set of tasks that covered the system's implementation were
formulated. These tasks were divided up into iterations and added to the
project planner on GitHub. Each task's dependencies were carefully considered
in order to prioritise tasks properly. The initial project plan can be seen in
Appendix \ref{sec:timeline}. 

\subsection{Support and Version Control Tools}

Partridge made heavy use of GitHub and the Git version control system. Git,
developed by Linus Torvalds, is a distributed Version Control System (VCS) that
offers many of the features of centralised VCSes such as Subversion and CVS.
However, unlike these centralised products, Git repositories are
self-contained, requiring no connection to a central server, with each copy of
the repository containing the complete project history. Copies of the
repositories are stored on the development machine, the production machine and
the GitHub server. This makes GIT more reliable than centralised repositories
since all code is accessible regardless of whether the repository hosted with
GitHub is online.

GitHub's hosted project management tools and wiki were used to store project
tasks, notes and metadata during development.

\section{Summary}

Having analysed some of the existing development methodologies and determined
that most of the common methodologies were not suitable for Partridge, a custom
methodology was formulated and support tools chosen. Task analysis was then
performed and a project timeline generated. With a timeline in place, the
initial design phase for Partridge and its subsystems could begin.
