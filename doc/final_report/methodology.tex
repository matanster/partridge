\subsection{Development Methodology}

\subsubsection{Existing Methodologies}
Selecting a suitable development methodology for building Partridge is another
very important choice for the project.

Under the traditional `Waterfall' Software Development model, Requirements
Gathering, Analysis, Program Design, Coding, Testing and Operations were all
defined as formal phases in the development cycle. There is little flexibility
other than moving back up the waterfall to rectify mistakes after
testing\cite{Royce:1987:MDL:41765.41801}. This model was very focused on
paperwork and bureaucracy, trying to maintain a paper trail and manage risk
through accountability \emph{(Ibed)}. This approach to software development is
very heavyweight and slow and often produced software that did not match the
users' needs as a result \cite{Boehm1988}.

As an alternative to the heavyweight Waterfall approach, Beck \emph{et al.} came up
with the principle of the Agile Manifesto, favouring a lightweight, responsive
development model over the heavyweight slow waterfall
system\cite{beck2001agile}. Many of Beck's ideas focus around working in a team
of developers and prioriting communication between team members \emph{(Ibid.)}.
This is most prominent in the Extreme Programming (XP) method of software
development. Since Partridge is an individual project, XP is not really
applicable. However, some concepts like rapid prototyping/spike work and
iterative release cycles will be used as part of the Partridge development
methodology.

\subsubsection{ Partridge's Development Methodology}

Partridge is an individual project but does involve discussions with
supervisors. The customers have been identified as the end-users of the system.
Therefore, a customised methodology has been adopted. Firstly, all design and
planning documentation have been written up and placed on a wiki which is
accessible and modifiable by the author and both supervisors. This creates a
paper trail for all tasks and also allows collaboration between involved
parties through the Internet. A full printout of the wiki is available in
Appendix \ref{sec:wiki}. 

Weekly meetings are held with both supervisors. The notes from the preceeding
week are analysed and each task discussed in depth. New tasks are then noted
down along with any observations that should be documented. These new notes are
uploaded to the wiki the following day or earlier. Each party present at the
meeting adds their own observations to the notes page. This page is then
reviewed at the next meeting. As seen in Appendix \ref{sec:wiki}, this practice
has already been running for several weeks and has so far proven to be highly
effective.

Partridge will adopt an Agile approach to release cycles, producing a working
software package at iterations of one month.  Each iteration, the software will
include more of the desired functionality discussed above and in the wiki.
GitHub's issue manager program is being used to track tasks and bugfixes and
plan which tasks will be carried out in which iteration. Tasks that are created
in a full iteration (where no development time is left) will be added to a
backlog and integrated in the next iteration with enough development time to
contain it. Tasks are also assigned a priority, higher priority issues being
tackled before low priority ones.

Partridge's testing strategy consists of multiple unit tests that are run at
integration of new code into the codebase. As soon as the first release is
built, Partridge will be made available for use by the public and users
encouraged to test the system and submit any bugs via the GitHub issue manager.
It is hoped that colleagues at Aberystwyth University and Dr. Liakata's
colleages at the EBI will try to use the system one it becomes available.

\subsubsection{ Work Timeline }

The tasks involved in Partridge have been carefully calculated and prioritised.
They were then added to the GitHub issue management system and a report
generated listing them in the order that they are expected to be
accomplished. This report can be seen in Appendix \ref{sec:timeline}. 


\subsubsection{ Mid-Project Demonstration Plan}

The mid-project demonstration is scheduled for after iteration two of the
Partridge project. If everything runs to schedule, then at this point it will
be possible to demonstrate keyword search within the project's database and
filter based upon the polarity of the paper's results. For redundancy purposes,
Partridge will be configured to run as a server on multiple computers. Both
this and the final project demonstration will require a room with Internet
access. However, should this be unavailable, then Partridge could be run
locally on the author's laptop.

\subsubsection{ Final Project Demonstration Plan}

The final demonstration of Partridge will be fairly similar to the Mid-Project
demonstration. However, it should include all of the planned classifiers and if
there is extra time on the project, the profiling/recommendation engine will
also be demonstrated. This demonstration will use the same redundancy
precautions as the Mid-project demonstration above, and will also need a room
with the internet if available.
