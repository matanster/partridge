%----------------------------------------------------------------
%
%  Listings of libraries used and where
%
%
%----------------------------------------------------------------

Below is a listing of third party libraries and frameworks used in Partridge, along with a
brief description of their functionality and their usage within the project.
Dependencies of the libraries are not listed, but can be found on the specific
library's home page

\section{Alembic}

\begin{tabular}{ | l | l | }

\hline
\textbf{ Home Page } &
\burl{https://bitbucket.org/zzzeek/alembic} \\

\textbf{ Author } & Mike Bayer \\

\textbf{ License } & MIT License \\
\hline

\end{tabular}

\subsection{Description}

Alembic is a database version control system that keeps production RDBMS
servers synchronised with the internal data structure of the application.

\subsection{Usage in Partridge}

Alembic was used to maintain the database at
\url{http://farnsworth.papro.org.uk} and keep its structure synchronised with
the development database as changes were made to Partridge.


\section{Bootstrap}

\begin{tabular}{ | l | l | }

\hline
\textbf{ Home Page } &
\burl{https://github.com/twitter/bootstrap} \\

\textbf{ Author } & Mark Otto \& Jacob Thornton \\

\textbf{ License } & Apache Public License \\
\hline

\end{tabular}

\subsection{Description}

Bootstrap is a design framework that uses CSS and Javascript to provide
boilerplate code and functionality so that developers don't have to `reinvent
the wheel'.

\subsection{Usage in Partridge}

Bootstrap was used to accelerate the development of the frontend interface by
providing a basic framework to build the interface on top of.


\section{CRFSuite}

\begin{tabular}{ | l | l | }

\hline
\textbf{ Home Page } &
\burl{http://www.chokkan.org/software/crfsuite/} \\

\textbf{ Author } & Naoaki Okazaki  \\

\textbf{ License } & Modified BSD License \\
\hline

\end{tabular}

\subsection{Description}

CRFSuite is an implementation of Conditional Random Fields, a technique used in
machine learning for labelling sequential features, rather than assuming that
features are conditionally independent as in Naive Bayesian Models.

\subsection{Usage in Partridge}

CRFSuite is used primarily in the Python version of SAPIENTA which I spent a
great deal of time working with and integrating, despite not using it in the
final submission of the project. I had to make some slight modifications to the
library to get it to compile and run as discussed in Section
\ref{sec:sapienta_implementation}. 

\section{Flask}

\begin{tabular}{ | l | l | }

\hline
\textbf{ Home Page } &
\burl{http://flask.pocoo.org/} \\

\textbf{ Author } & Armin Ronacher \\

\textbf{ License } & BSD License \\
\hline

\end{tabular}

\subsection{Description}

Flask is a web framework and utility library for the development of Python web
applications. It provides wrappers for low level TCP communication and templating
functionality from within a model-view-controller architecture.

\subsection{Usage in Partridge}
Flask is used extensively in Partridge as a platform for serving the paper
content and interfaces to the user web browser. The templating library is used
to insert data into the user interface at run time and the JSON encoder is used
to provide communication between the JavaScript running in the user's browser
and the web backend.


\section{Flask-SQLAlchemy}

\begin{tabular}{ | l | l | }

\hline
\textbf{ Home Page } &
\burl{http://pythonhosted.org/Flask-SQLAlchemy/} \\

\textbf{ Author } & Armin Ronacher \\

\textbf{ License } & BSD License \\
\hline

\end{tabular}

\subsection{Description}
Flask-SQLAlchemy provides a bridge between Flask-based applications and the
SQLAlchemy library, it is quite lightweight and mainly used to prevent
reimplementation of boilerplate database connection code.

\subsection{Usage in Partridge}

Flask-SQLAlchemy is used in Partridge to establish a connection to the SQL
database used to store the paper data and to provide a convenient way to access
the database from the preprocessor and testing utilities.

\section{jQuery}

\begin{tabular}{ | l | l | }

\hline
\textbf{ Home Page } &
\burl{http://jquery.com} \\

\textbf{ Author } & jQuery Foundation\\

\textbf{ License } & MIT License \\
\hline

\end{tabular}

\subsection{Description}

jQuery is a javascript library containing sets of utilities for manipulating
HTML web pages and managing Javascript HTTP requests to web servers. It
assists in helping developers write cross-platform compatible, stable code.

\subsection{Usage in Partridge}

jQuery is used extensively in Partridge's frontend. It is used for manipulating
the user interfaces once they have been loaded and rendered in the web browser
and to facilitate browser-server calls during search queries and file uploads.


\section{jQuery-BBQ}

\begin{tabular}{ | l | l | }

\hline
\textbf{ Home Page } &
\burl{http://benalman.com/projects/jquery-bbq-plugin/} \\

\textbf{ Author } & Ben Alman \\

\textbf{ License } & Dual License: MIT and GPL \\
\hline

\end{tabular}

\subsection{Description}

jQuery-BBQ is a plugin to the jQuery library that facilitates the manipulation
of a web page's URI fragment through the use of JavaScript. This allows pages
that carry out a lot of clientside processing to store the application state in
the fragment and use the Browser History for navigating through changes.

\subsection{Usage in Partridge}

Partridge uses jQuery-BBQ to serialise paper queries into the URI fragment of
the query page and therefore share query results and use their browser history
to undo query constraints.


\section{jQuery-HTML5-Upload}

\begin{tabular}{ | l | l | }

\hline
\textbf{ Home Page } &
\burl{https://github.com/mihaild/jquery-html5-upload} \\

\textbf{ Author } & Mikhail Dektyarev \\

\textbf{ License } & Unspecified/Public Domain \\
\hline

\end{tabular}

\subsection{Description}

jQuery HTML5 Upload is a small plugin to the jQuery library that provides a
simple JavaScript interface for asynchronously uploading files using HTML POST
requests and providing progress notifications.

\subsection{Usage in Partridge}

This library is used in the Partridge Paper Uploader view for sending papers to
the server and the progress notifier used to set up a progress bar to inform
the user of how many of the papers have been uploaded.


\section{MatPlotLib}

\begin{tabular}{ | l | l | }

\hline
\textbf{ Home Page } &
\burl{http://matplotlib.org/} \\

\textbf{ Author } & John Hunter \\

\textbf{ License } & Python Software Foundation License \\
\hline

\end{tabular}

\subsection{Description}

MatPlotLib is a library for rendering 2D and 3D graphs from the Python
programming language. It provides a simple programming interface that allows
the generation of a wide number of different types of graph and also provides
functions for exporting the graphs as image files and storing them on the
user's hard disk.

\subsection{Usage in Partridge}

MatPlotLib was used during the implementation and testing of Partridge,
specifically when working on the Machine Learning elements of the system. It
was not used in the live Partridge system, but it was used in some of the test
scripts. Some of the graphics included in this dissertation were generated with
MatPlotLib.


\section{NLTK}

\begin{tabular}{ | l | l | }

\hline
\textbf{ Home Page } &
\burl{http://nltk.org/} \\

\textbf{ Author } & Steven Bird, Edward Loper and Ewan Klein\\

\textbf{ License } & Apache Public License \\
\hline

\end{tabular}

\subsection{Description}

NLTK or the Natural Language Toolkit is a library for Python that provides
implementations of commonly used Natural Language Processing tools and
utilities. NLTK comes with sets of test data and pre-trained models for
convenience.

\subsection{Usage in Partridge}

The NLTK is used in Partridge's preprocessor during the Sentence Splitting
stage. It is not used in any of the other subsystems, but would have been used
to implement the Paper Topic and Paper Result classifiers had there been time.


\section{Orange}

\begin{tabular}{ | l | l | }

\hline
\textbf{ Home Page } &
\burl{http://orange.biolab.si/} \\

\textbf{ Author } & Tomaž Curk et al. \\

\textbf{ License } & GNU General Public License \\
\hline

\end{tabular}

\subsection{Description}

Orange is a data mining library for Python, providing implementations of
commonly used machine learning algorithms. The underlying implementations are
written using native code to increase the speed of the algorithms.

\subsection{Usage in Partridge}

Orange's Random Forest Classifier is used to provide paper type classification
within the preprocessor module. It was also used for clustering the papers
during the investigation into the most effective features for classification.


\section{PDFX}

\begin{tabular}{ | l | l | }

\hline
\textbf{ Home Page } &
\burl{http://pdfx.cs.man.ac.uk/} \\

\textbf{ Author } & Alex Constantin \\

\textbf{ License } & Other \\
\hline

\end{tabular}

\subsection{Description}
PDFX is a web service that converts PDF documents into SciXML compatible
markup. The author provides a REST API which is documented on the home page.
No code is provided, all computations are executed by the server.

\subsection{Usage in Partridge}

PDFX is used by the paper preprocessor to convert uploaded PDF documents into
XML for SAPIENTA to annotate. The PDFs are uploaded to the PDFX server using
HTTP requests.

\section{SAPIENTA}

\begin{tabular}{ | l | l | }

\hline
\textbf{ Home Page } &
\burl{http://www.sapientaproject.com/} \\

\textbf{ Author } & Maria Liakata \\

\textbf{ License } & Other \\
\hline

\end{tabular}

\subsection{Description}
SAPIENTA is a machine learning application that provides CoreSC annotations for
papers formatted in SciXML and PubMed Central compatible markup. Although a
SAPIENTA Python library is available, it is not ready for general use, and the
web service is more reliable for paper annotation.

\subsection{Usage in Partridge}

SAPIENTA is key in paper processing. It is used to annotate the papers stored
in Partridge using CoreSC labels. Like PDFX, no SAPIENTA code is used on the
client side and communication with the remote system is carried out through
HTTP requests.


\section{SQLAlchemy}

\begin{tabular}{ | l | l | }

\hline
\textbf{ Home Page } &
\burl{http://www.sqlalchemy.org/} \\

\textbf{ Author } & Michael Bayer \\

\textbf{ License } & MIT License \\
\hline

\end{tabular}

\subsection{Description}
SQLAlchemy provides a set of functions and utilities for connecting to and
communicating with SQL database systems from the Python programming language.
It also offers an extensive Object Relation Mapper which allows the abstraction
of database entities into Python Data structures for the convenience of
developers.

\subsection{Usage in Partridge}

SQLAlchemy is used by Partridge to communicate with the paper database. A set
of data structures that derrive properties provided by this library allow
Partridge to use native Python objects to refer to entities in the database.
SQLAlchemy also carries out data encoding conversion and input sanitisation
capabilities transparently.
