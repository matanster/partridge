\subsection{Meeting Summary 10/10/2012}

\subsubsection{From Maria's notes}

\begin{itemize}
\item
  James has gone through papers and thinks would be more doable to work
  with papers than Books
\item
  James has found the XML format of PlosOne and ART/CoreSC corpus
\item
  Domain needs to be determined
\item
  Maria to send paper on sentiment analysis of citations
\end{itemize}

\subsubsection{Current features and recommendations}

\begin{itemize}
\item
  Analysis of results section - positive or negative result
\item
  Potentially look at writing styles of authors using syntactic analysis
\item
  Isolating terms within sections of papers (i.e.~find me papers where
  methodology x was used)
\item
  finer grain analysis of CoreSC categories
\item
  nltk python toolkit. What other toolkits? Would jerboa be any good?
\item
  common features for sentiment analysis
\end{itemize}

\subsubsection{Basic system design}

I have started a basic system design although it is very basic at the
moment. [Image Omitted]

\subsubsection{Background - Similar Systems and how they work}

\begin{itemize}
\item
  Google Scholar
\item
  Mendeley
\item
  Citeulike
\item
  Tweeting or `liking' on facebook.
\end{itemize}

\subsubsection{Reading}

\begin{itemize}
\item
  TF-IDF
\item
  Take a brief look at journal of negative results
\item
  `Bigrams + trigrams' and syntactic pattern finding
\item
  Common features used in NLP
\item
  Sentiment analysis of citations in papers
\item
  `Bag of words'
\end{itemize}

\subsubsection{Coding/practical research}

\begin{itemize}
\item
  Play some more with SAPIENTA
\item
  Implement a (La)TeX to SciXML or pubmed command-line tool.
\end{itemize}
